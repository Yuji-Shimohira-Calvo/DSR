\documentclass{article}
\usepackage[english]{babel}
\usepackage[utf8]{inputenc}
\usepackage{fancyhdr}
\usepackage{graphicx}
\usepackage{float}
\usepackage{listings}
\usepackage[parfill]{parskip}
\usepackage{fourier}

\pagestyle{fancy}
\fancyhf{}
\rhead{Proudly \LaTeX}
\lhead{Doing Survey Research 2018}
\rfoot{Page \thepage}

\newcommand{\forceindent}{\leavevmode{\parindent=2em\indent}}
 
\begin{document}
	
\section*{\hfil Lab Worksheet VIII \hfil}
\subsection*{Simple regression analysis}

This lab worksheet guides you through the basics of doing simple regression analysis in Stata. The exercise does not really teach you the theory behind regression analysis, instead it focuses on the technical aspect of doing with Stata what you learned in your lectures. If you feel you need a reminder you can look up in the course's materials what regression analysis, regression coefficients, and $R^2$ are. It is also fundamental to know the difference between dependent and independent variables (the latter are also called \textit{predictors} or \textit{regressors}).

The data we are going to use come from the European Social Survey Round 8 (2016) in Great Britain. The dataset is called and you can find it on LEARN. As you already know, the first step when dealing with new data is to explore them. You can do this by using the \texttt{describe}, \texttt{summarize}, and \texttt{codebook} commands.

 

\end{document}