%---------------------------
%	PREAMBLE
%---------------------------

\documentclass{article}

\usepackage[english]{babel}
\usepackage[utf8]{inputenc}
\usepackage{fourier}
\usepackage{fancyhdr}
\usepackage[parfill]{parskip}
\usepackage{hyperref}
\usepackage{graphicx}
\usepackage{float}
\usepackage{listings}
\usepackage{fourier}
\usepackage{tikz}
\usetikzlibrary{shapes.geometric, arrows}
\usepackage{textcomp}

\newcommand{\forceindent}{\leavevmode{\parindent=2em\indent}}

%------------------------%------------------------%

\begin{document}

\pagestyle{fancy}
\fancyhf{}
\rhead{Doing Survey Research \the\year}
\lhead{https://github.com/Yuji-Shimohira-Calvo/DSR}
\rfoot{Page \thepage}

\section*{\hfil Lab Worksheet I \hfil}
\subsection*{Getting data}

Let's start by opening the following URL: \url{http://ukdataservice.ac.uk}. The \textbf{UK Data Service (UKDS)} offers a single point of access to a vast number of secondary data ranging from qualitative data to international macro-data. This comprehensive archive of high-quality data is funded by the \textbf{Economic and Social Research Council (ESRC)}. For this first lab session we are going to use data coming from the \textbf{European Social Survey (ESS)} 2012 for the United Kingdom.

\begin{itemize}
	\item Click on the \textbf{Get Data} tab situated at the central panel of the website.
	\item You could now type ``european social survey'' in the search box. However, you are going to use instead the \textbf{Quick Access} side panel. Now click on \textbf{Cross-national surveys} and scroll down to ``European Social Survey.'' Click on it.
\end{itemize}

What you see now is a page that shows basic information about all rounds of the ESS. For instance, under ``Title Details'' you can see that the ESS project is funded by the Commission of the European Communities, the European Science Foundation, and other national funding bodies of Europe. If you scroll down you will find information about subject categories, main topics of each round, and how to access the data among many other useful things. Let's clink now on ``Access online'' (marked with a white star on a blue background at the top of the page). You should now at the website of the ESS project, where you can easily navigate through one tab to another. These tabs offer well-structured information on everything you need to know about the ESS. Under \textbf{About ESS} you will find institutional information like how the project is funded, what countries participate in the making of the ESS, and so on. The \textbf{Methodology} tab presents technical details like sampling strategy, pre-testing, piloting, and so on. \textbf{Data and Documentation} allows you to navigate through all rounds of the ESS in three different ways: by year, by country, and by theme. Let's now take a look at the questionnaire used in the United Kingdom in 2012 (sixth round):

\begin{itemize}
	\item Mouse hover over \textbf{Data and Documentation} and click on \textbf{Data and Documentation by Country}. This will show you a list of all countries that have taken part in the ESS.
	\item Scroll down and click on ``United Kingdom.'' Under \textbf{Documents} click on ``ESS6 Questionnaires GB'' (you can download a PDF file of the questionnaire or read it with your browser's built-in PDF reader). Keep the questionnaire visible in order to answer the questions below.
\end{itemize}

Take a look at the introduction (page 3). Please write down your answers in the boxes situated below each question.

\pagebreak

\forceindent \textbf{?} The introduction seems to be structured around three elements. How would name each of them? Why do think the introduction is structured in this way?

\framebox(347,100){}

Let's move on to ``Section B'', which is about politics and governance. Note that all questions are named first with a letter corresponding with the section they belong to (B in this case) followed by a number corresponding to the order of that question within its section.

\forceindent \textbf{?} Looking at questions B11-B17, what do you think they are trying to measure here? Do you think the list of responses is comprehensive? Why/why not?

\framebox(347,100){}

\forceindent \textbf{?} Try to identify in the questionnaire the sociodemographic questions. In which section are they? What kind of characteristics do they ask about? Why do you think these questions are situated towards the end of the questionnaire rather than at the beginning?

\framebox(347,100){}

\pagebreak

\forceindent \textbf{?} Go now to page 66. The questionnaire says that, in order to improve the questionnaire in the future, the interviewer will ask some final questions about similar topics that the interviewee has already responded to. How do you think this helps the designers of the questionnaire? Also, find the equivalent questions for IF13, IF14, and IF15, and jot down concisely what has changed.

\framebox(347,100){}

The UKDS also allows you to run a search by topic or theme. Imagine you are interested in exploring political participation. The ESS provides a considerable range of questions on political participation, but you may be interested in seeing how different surveys approach it (this is a good way to critically engage with a topic). For this, let's go back to the UKDS webpage (\url{http://ukdataservice.ac.uk}) and in the search box situated to the right type ``political participation'' (make sure ``Data'' is selected and not ``Website'').

As you can see 822 results show up. You can sort them by relevance, date, title, and downloads. If you sort the results by ``most downloaded data'' you will see that the first result to appear (at the time of writing) is the \textbf{British Household Panel Survey (BHPS)}. The BHPS is a survey that, like the ESS, covers a broad range of different topics. This time we are interested only in political participation, so let's sort the results again by ``relevance'' and click on ``Audit of Political Engagement 9, 2011'' (APE9). Moreover, the UKDS website also has a search-box in which you can type variable names (keep in mind that in a survey questionnaire a question is an \textit{operationalisation} of a variable). You can find this ``variable search-box'' at \url{http://discover.ukdataservice.ac.uk/variables}, but you can also get to it by clicking on the \textbf{Get Data} tab and then click again on the box situated to the right (which says ``variable and question bank''). For the time being we are going to work with the Audit of Political Engagement 9, 2011 survey. Now try to answer the following questions:

\pagebreak

\forceindent \textbf{?} Get a copy of the questionnaire and jot down the main differences you see when you compare it to the ESS questionnaire (keep it open too).

\framebox(347,100){}

\forceindent \textbf{?} Perhaps you have noticed that questions Q4A and Q5 (APE9 questionnaire) ask respondents about a series of things that they did (or did not) do in the last two or three years. On the other hand, the ESS asks only about actions performed (or not performed) in the last twelve months. Write in the box below whether you would choose one or the other and the reasons why.

\framebox(347,100){}

\forceindent \textbf{?} Take a look at questions Q5 (APE9) and B11-B17(ESS6). What other differences do you see in their phrasing? Which one would you choose and why? Also, would it make sense to operationalise political participation with a ranking scale? Why/why not?

\framebox(347,100){}

\end{document}
